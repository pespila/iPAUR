\section{The Mumford-Shah Functional} % (fold)
\label{sec:the_mumford_shah_functional}

	\subsection{The Mumford-Shah Model as Saddle-Point Problem} % (fold)
	\label{sub:the_mumford_shah_model_as_saddle_point_problem}
	
		We already saw the formulation of the saddle-point problem in the section before. Recalling equation \ref{eq:mumford_shah_saddle_point_problem} and setting $K = \nabla$ we have that a minimizer of the Mumford-Shah Functional can be computed by solving
			\begin{equation}
				\min_{u \in C} \max_{p \in K} \langle \nabla u, p \rangle.
				\label{eq:primal_dual_mumford_shah}
			\end{equation}

		As in the chapter before we first want to formulate this saddle-point problem in the primal and the dual version. Further, we want to bring this primal-dual version into a setting, where we can make use of one of our algorithms. At the end, we are able to compute the proximity operators and for that we give some formulas to implement the algorithm.

		We can write equation \ref{eq:primal_dual_mumford_shah} in a slightly different way, since we can simply add the indicator functions of the corresponding sets $C$ and $K$. We observe
			\begin{equation}
				\min_{u \in C} \max_{p \in K} \langle \nabla u, p \rangle - \delta_{K}(p) + \delta_{C}(u).
				\label{eq:primal_dual_mumford_shah_complete}
			\end{equation}
		But then, we immediately can determine our functions $F^{\ast}$ and $G$, which are given by
			$$
				F^{\ast}(p) = \delta_{K}(p) \,\,\,\,\, G(u) = \delta_{C}(u).
			$$
		With this we first want to take a look at the dual formulation of the Mumford-Shah model. The dual of the saddle-point problem was given by
			$$
	            \max_{p \in Y} -(G^{\ast}(-K^{\ast}p) + F^{\ast}(p)).
	        $$
	    We need to compute the Legendre-Fenchel conjugate of $G$. Since this is the indicator function of the set $C$ we can make use of Example \ref{ex:legendre_fenchel_conjugate_example} 1. and see that $G^{\ast}(p) = \delta^{\ast}_{C}(p) = \sup_{u \in C} \langle p, u \rangle$, which is the support function of $C$. Overall, we obtain the dual Mumford-Shah problem by
	    	\begin{equation}
	    		\max_{p \in Y} -(G^{\ast}(-K^{\ast}p) + F^{\ast}(p)) = \max_{p \in K \subset Y} -(\langle \nabla^{T}p, u \rangle + \delta_{K}(p)) = \max_{p \in K} -\langle \nabla^{T}p, u \rangle.
	    	\label{eq:dual_mumford_shah}
	    	\end{equation}
	    Finally, we also want to evaluate the primal formulation of the Mumford-Shah saddle-point problem. Therefore, it is left to compute $F$ from $F^{\ast}$. Assume we have a function $F(\nabla u)$, then by definition the Legendre-Fenchel conjugate is given by
	    	$$
	    		F^{\ast}(p) = \sup_{u \in C} \langle \nabla u, p \rangle - F(\nabla u).
	    	$$
	    On the other hand, if we assume $F$ to be convex we observe
	    	$$
	    		F(\nabla u) = \sup_{p \in K} \langle \nabla u, p \rangle - F^{\ast}(p) = \sup_{p \in K} \langle \nabla u, p \rangle - \delta_{K}(p)
	    	$$
	    which is exactly what we have in equation \ref{eq:primal_dual_mumford_shah_complete}. So, we want to compute $F(\nabla u)$ and show that the function $F$ is indeed convex. Computing $F$ means solving
	    	$$
	    		\sup_{p \in K} \langle \nabla u, p \rangle - \delta_{K}(p) = \sup_{p \in K} \langle \nabla u, p \rangle.
	    	$$
	    We drop the indicator function, because in the case $p \notin K$ the function would be $\delta_{K}(p) = \infty$, and for that we would not attain the supremum. Also note that we are seeking for the supremum for all $p \in K$, for that the indicator function is zero.
	    Back to our function $F$, we see that since $K$ is finite dimensional, we can change the supremum to the maximum. We observe
	    	$$
	    		\max_{p \in K} \langle \nabla u, p \rangle \Longleftrightarrow \nabla \big( \nabla u, p \big) = 0 \Longleftrightarrow \nabla u = 0.
	    	$$
	    But this means nothing that if the gradient of $u$ vanishes, $F(\nabla u) = 0$. If the gradient is less of greater than zero the only choice for $F$ over all $p$ can only be $F(\nabla u) = \infty$. This holds of course for an arbitray argument of $F$, so that we have for a $u \in C$
	    	$$
	    		F(u) =
	    			\begin{dcases*}
	    				0 & \textnormal{if $u = 0$,} \\
	    				\infty & \textnormal{else.}
	    			\end{dcases*}
	    	$$
    	This is nothing but the indicator function of a single point. Then for the primal formulation we get
	    	\begin{equation}
	    		\min_{u \in C} F(\nabla u) + G(u) = \min_{u \in C} G(0) = 0,
	    		\label{eq:primal_mumford_shah}
	    	\end{equation}
	    because we know from the definition of the set $C$ that $0 \in C$. To solve the primal formulation one only needs to solve a linear equation, namely $\nabla u = 0$, meaning, that the minimum can only be attained if $F(\nabla u) = 0$.

	    % The question we need to answer is, for what this formulation here is useful for. We earlier introduced the Primal-Dual Gap. And we saw that this gap vanishes if and only if $(u, p)$ are saddle-points. In other words, if we let our iterations $n$ go to $\infty$, then
	    % 	$$
	    % 		\max_{\tilde{p} \in Y} \langle \tilde{p}, \nabla u \rangle - F^{\ast}(\tilde{p}) + G(u) - \min_{\tilde{u} \in X} \langle p, \nabla \tilde{u} \rangle - F^{\ast}(p) + G(\tilde{u}) = 0.
	    % 	$$
	    % To compute the Primal-Dual Gap in each iteration step, we need to solve the primal and the dual formulation ,respectively. 

	    % We are set up to to compute the proximity operators of the Mumford-Shah model.

   % subsection the_mumford_shah_model_as_saddle_point_problem (end)

    \subsection{The Proximity Operators for the Mumford-Shah Model} % (fold)
    \label{sub:the_proximity_operators_for_the_mumford_shah_model}
    	
    	In this subsection we will see, that for the Mumford-Shah model we need to compute projections onto convex sets. This will then be discussed in the next section. But first let us note that $F^{\ast}(p) = \delta_{K}(p)$ and $G(u) = \delta_{C}(u)$. From example \ref{ex:proximity_operator} we know that the proximity operator of the indicator function is a projection onto the corresponding convex set. This implies, that the proximity operators for $F^{\ast}$ and $G$ are projections onto $K$ and $C$, respectively. For that we want to rewrite our primal-dual algorithm to be consistent with \cite{Pock-et-al-iccv09}. We observe

    		\begin{algorithm}\label{alg:primal_dual_cremers}
                Choose $(x^{0}, y^{0}) \in C \times K$ and let $\bar{x}^{0} = x^{0}$. We choose $\tau, \sigma > 0$. Then, we let for each $n \ge 0$
                    \begin{equation}
                        \left\{ 
                            \begin{array}{l l}
                              y^{n+1} = \Pi_{K} (y^{n} + \sigma K \bar{x}^{n}) \\
                              x^{n+1} = \Pi_{C} (x^{n} - \tau K^{*} y^{n+1}) \\
                              \bar{x}^{n+1} = 2x^{n+1} - x^{n}.
                            \end{array}
                        \right.
                    \end{equation}
            \end{algorithm}

        Here we denote $\Pi_{K}$ and $\Pi_{C}$ as the (euclidean) projections on the sets $K$ and $C$. How projections onto these two convex sets are derived will be discussed in the next section.
        
    % subsection the_proximity_operators_for_the_mumford_shah_model (end)
% section the_mumford_shah_functional (end)